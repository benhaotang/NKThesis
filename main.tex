%%%%%%%%%%%%%%%%%%%%%%%%%%%%%%%%%%%%%%%%%%%%%%%%%%%%%%%%%%%%%%%%%%%%%%%%%
%%
%%            南开大学毕业论文模板
%%                Ver 211214.2
%%         原模板作者:数院孙文昌教授
%%        物院本地化&新版本维护:唐本豪
%%         (benhaotang@outlook.com)
%%%%%%%%%%%%%%%%%%%%%%%%%%%%%%%%%%%%%%%%%%%%%%%%%%%%%%%%%%%%%%%%%%%%%%%%%
%%使用注意:
%%  本模板推荐使用XeLaTeX
%%
%%  注意:
%%    1. 在改变编译方式前应先删除 *.toc 和 *.aux 文件,
%%       因为不同编译方式产生的辅助文件格式可能并不相同。
%%    2. 引用请修改nkthesis.bib并采用bibtex
%%    3. 为了减少不必要的麻烦,请把正文写在mainbody.tex中
%%
%%  本模板原始来自于数学科学学院孙文昌教授按照2017年南开大学毕业论文要求编写,
%%  感谢孙老师完成了整个模板的标准格式编写
%%
%%%%%%%%%%%%%%%%%%%%%%%%%%%%%%%%%%%%%%%%%%%%%%%%%%%%%%%%%%%%%%%%%%%%%%%%%
\documentclass[12pt,openright]{book}
\usepackage{xifthen}
\usepackage{ifxetex}
\ifxetex
  \usepackage[bookmarksnumbered]{hyperref}
\else
  \usepackage[unicode,bookmarksnumbered]{hyperref}
\fi
\usepackage[emptydoublepage]{NKThesis}   % 中文
%\usepackage[emptydoublepage,English]{NKThesis} % 英文
\usepackage{amssymb}
%   根据需要选择 biblatex 宏包选项.
\usepackage[
backend = biber, style = gbt77142005, utf8
]{biblatex}%211214 这里引用了国标的标准引用格式,为了正常使用,建议不要再更改任何与bibtex相关的设置了。另外这里虽然是借鉴了github上的那个项目,但是由于兼容性问题已经把github上原来的代码不兼容的地方改掉了,请不要再去百度上搜索再重新覆盖了,出现编译问题我没办法再帮助您了。
\hypersetup{colorlinks=true,
            pdfborder=0 0 1,
            citecolor=black,
            linkcolor=black}
\usepackage{tikz}
\usepackage{amsmath}
\DeclareSymbolFont{epsilon}{OML}{ntxmi}{m}{it}
\DeclareMathSymbol{\epsilon}{\mathord}{epsilon}{"0F}%追加对于varepsilon支持 211214
\addbibresource{nkthesis.bib}
\DeclareBibliographyCategory{cited}
\AtEveryCitekey{\addtocategory{cited}{\thefield{entrykey}}}
\includeonly{
abstract,
mainbody,
acknowledgements,
references,
appendices,
resume
}
\newtheorem{Theorem}{\hskip 2em 定理}[chapter]
\newtheorem{Lemma}[Theorem]{\hskip 2em 引理}
\newtheorem{Corollary}[Theorem]{\hskip 2em 推论}
\newtheorem{Proposition}[Theorem]{\hskip 2em 命题}
\newtheorem{Definition}[Theorem]{\hskip 2em 定义}
\newtheorem{Example}[Theorem]{\hskip 2em 例}
\newcommand{\upcite}[1]{\textsuperscript{\textsuperscript{\cite{#1}}}}
\begin{document}

%  设置基本信息
%  注意:  逗号`,'是项目分隔符. 如果某一项的值出现逗号, 应放在花括号内, 如 {,}
%
\NKTsetup{%
  论文题目(中文) = 论文中文题目,
  副标题         = 论文中文副标题,
  论文题目(英文) =  English Title,
  论文作者       = 张十三,
  学号           = 1810xxx,
  指导教师       = 张三\quad 教授,
  申请学位       = 理学学士,
  培养单位       = 物理科学学院,
  学科专业       = 某某某某,
  研究方向       = 某某某某,
  中图分类号     = ,
  UDC            = ,
  学校代码       = 10055,
  密级           = 公开,
                   % 公开 | 限制 | 秘密 | 机密, 若为公开, 不填以下三项
  保密期限       = ,
  审批表编号     = ,
  批准日期       = ,
  论文完成时间   = 二〇二一年十一月,
  答辩日期       = ,
  论文类别       = 本科,
                   % 本科 | 博士 | 学历硕士 | 硕士专业学位 | 高校教师 | 同等学力硕士
                   % 211102 追加了本科学士学位,去除了高校教师的选项
  学院(单位)       = 物理科学学院,
  答辩委员会主席       = 李四,
  评阅人       = 李四\quad 王五,
  专业           = 某某某某,
  联系电话       = 1234567,
  Email          = 1810xxx@nankai.edu.cn,
  通讯地址(邮编) = 天津市南开区卫津路94号(300071),
  备注           = }


% -*- coding: utf-8 -*-


\begin{zhaiyao}
请在这里写摘要
\end{zhaiyao}




\begin{guanjianci}
摘要;Abstract
\end{guanjianci}



\begin{abstract}
English Abstract Here!

\end{abstract}



\begin{keywords}
Abstract; \LaTeX
\end{keywords} 
\tableofcontents
\include{mainbody}
% -*- coding: utf-8 -*-
\printbibliography[category = cited,title={参考文献}]



\include{acknowledgements}
\begin{Appendix}
      \section{在这里插入附录}
      通过Appendix环境插入附录。
\end{Appendix}
% -*- coding: utf-8 -*-


\chapter*{个人简历}

\noindent 姓名:张十三\\
出生日期:2000年1月1日\\
\\
教育背景:\\
2018年9月-2022年7月\quad 南开大学\quad 物理科学学院\quad\quad\quad\quad 某某学\quad\quad 学士 \\
\\
本科期间发表的学术论文:\\

\end{document}
